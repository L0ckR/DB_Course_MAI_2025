% Compile with XeLaTeX or LuaLaTeX to use Times New Roman
\documentclass[14pt,a4paper]{extarticle}

\usepackage{fontspec}
\usepackage{polyglossia}
\setdefaultlanguage{russian}

% Prefer Times New Roman for ГОСТ requirements; fall back if the font is missing.
\IfFontExistsTF{Times New Roman}{
  \setmainfont{Times New Roman}
  \newfontfamily\cyrillicfont{Times New Roman}
}{
  \setmainfont{TeX Gyre Termes}
  \IfFontExistsTF{PT Serif}{
    \newfontfamily\cyrillicfont{PT Serif}
  }{
    \newfontfamily\cyrillicfont{DejaVu Serif}
  }
}
\usepackage[left=30mm,right=15mm,top=20mm,bottom=20mm]{geometry}
\usepackage{setspace}
\onehalfspacing
\usepackage{indentfirst}
\usepackage{graphicx}
\usepackage{longtable}
\usepackage{hyperref}
\usepackage{listings}
\usepackage{enumitem}

\lstset{
  basicstyle=\small\ttfamily,
  breaklines=true,
  frame=single,
  columns=fullflexible
}

\begin{document}

\begin{titlepage}
  \begin{center}
    \textbf{Название образовательной организации}\\
    \textbf{Факультет / институт}\\
    \textbf{Кафедра}\\[2cm]

    \textbf{Пояснительная записка к курсовой работе}\\
    по дисциплине «Базы данных»\\[1.5cm]

    \textbf{Тема:} Информационная система для управления ML‑экспериментами и моделями\\[2cm]
  \end{center}

  \vfill

  \begin{flushright}
    Выполнил: \underline{\hspace{6cm}}\\
    Группа: \underline{\hspace{6cm}}\\[0.5cm]
    Руководитель: \underline{\hspace{6cm}}\\
  \end{flushright}

  \vfill

  \begin{center}
    Город, \the\year\ г.
  \end{center}
\end{titlepage}

\tableofcontents
\newpage

\section*{Введение}
\addcontentsline{toc}{section}{Введение}
Объект исследования — информационная система для управления жизненным циклом ML‑экспериментов и моделей.
Предмет исследования — проектирование реляционной базы данных, поддерживающей аудит, аналитические запросы
и интеграцию с backend‑API. Цель работы — разработать систему, демонстрирующую проектирование БД,
ограничения целостности, функции, триггеры, оптимизацию запросов и batch‑import.

\section{Аналитическая часть}
Системы типа MLflow/Weights\&Biases фиксируют конфигурации запусков, метрики и артефакты.
В курсовой работе важно обеспечить целостность данных и воспроизводимость, а также поддержать аналитические отчёты.
Ключевые сценарии: регистрация пользователей, создание проектов и экспериментов, запуск runs, логирование метрик,
хранение ссылок на артефакты и сравнение результатов.

\section{Проектная часть}
\subsection{Архитектура системы}
Архитектура включает PostgreSQL 16, backend на FastAPI + SQLAlchemy + Alembic и frontend для демонстрации.
Компоненты запускаются через Docker Compose.

\subsection{Проектирование структуры базы данных}
Схема включает 18 таблиц, обеспечивающих связи 1:1, 1:N и N:M:
\begin{itemize}[leftmargin=1.25cm]
  \item users, organizations, org\_members
  \item ml\_projects, project\_members
  \item datasets, dataset\_versions
  \item experiments, runs, run\_configs
  \item metric\_definitions, run\_metric\_values
  \item artifacts, run\_artifacts
  \item audit\_log
  \item batch\_import\_jobs, batch\_import\_errors
  \item project\_metric\_summary
\end{itemize}

Ключевые связи:
\begin{itemize}[leftmargin=1.25cm]
  \item 1:1 — runs → run\_configs.
  \item 1:N — projects → datasets; experiments → runs.
  \item N:M — runs ↔ artifacts (через run\_artifacts).
\end{itemize}

Ограничения целостности реализованы через PK, FK, UNIQUE, CHECK и NOT NULL, с каскадным удалением/обновлением
для зависимых сущностей. Поля времени: created\_at / updated\_at (где применимо).

\subsection{Представления}
Реализованы три VIEW:
\begin{itemize}[leftmargin=1.25cm]
  \item v\_runs\_with\_final\_metrics — финальные метрики по runs;
  \item v\_best\_runs\_per\_experiment — лучший run по ключевой метрике;
  \item v\_project\_quality\_dashboard — агрегаты по проекту (success rate, медиана времени, best metric).
\end{itemize}

\subsection{Функции и триггеры}
\textbf{Функции:}
\begin{itemize}[leftmargin=1.25cm]
  \item fn\_best\_run\_id — скалярная функция, возвращающая лучший run по метрике и цели (min/max/last).
  \item fn\_experiment\_leaderboard — табличная функция для top‑N runs по метрике.
\end{itemize}

\textbf{Триггеры:}
\begin{itemize}[leftmargin=1.25cm]
  \item fn\_audit\_log — аудит INSERT/UPDATE/DELETE для ключевых таблиц.
  \item fn\_update\_project\_metric\_summary — поддержка агрегатов в project\_metric\_summary.
\end{itemize}

\subsection{API и взаимодействие с БД}
API предоставляет CRUD для всех сущностей и отчётные endpoints.
Отчёты вызывают функции и представления (leaderboard, best‑run, dashboard). Batch‑import логирует ошибки
в batch\_import\_errors и сохраняет статистику в batch\_import\_jobs.

\subsection{Примеры бизнес‑запросов}
Ниже приведены примеры SQL‑запросов из \texttt{sql/business\_queries.sql}.

\begin{lstlisting}[language=SQL,caption={Лидерборд по эксперименту с дельтой к среднему}]
WITH final_metrics AS (
    SELECT r.experiment_id, r.run_id, r.run_name, rmv.value, md.goal,
           ROW_NUMBER() OVER (PARTITION BY r.experiment_id
             ORDER BY CASE WHEN md.goal='min' THEN rmv.value END ASC,
                      CASE WHEN md.goal='max' THEN rmv.value END DESC,
                      CASE WHEN md.goal='last' THEN rmv.recorded_at END DESC) AS rn,
           AVG(rmv.value) OVER (PARTITION BY r.experiment_id) AS avg_value
    FROM runs r
    JOIN run_metric_values rmv ON rmv.run_id = r.run_id AND rmv.step IS NULL
    JOIN metric_definitions md ON md.metric_id = rmv.metric_id
    WHERE md.key = :metric_key AND rmv.scope = :scope
)
SELECT experiment_id, run_id, run_name, value AS best_value, avg_value,
       value - avg_value AS delta_vs_avg
FROM final_metrics
WHERE rn = 1;
\end{lstlisting}

\section{Технологическая часть}
\subsection{Контейнеризация и запуск}
Проект разворачивается через Docker Compose. Backend автоматически применяет миграции Alembic при старте.
Последовательность запуска:
\begin{enumerate}[leftmargin=1.25cm]
  \item \texttt{docker compose up --build}
  \item \texttt{docker compose exec backend python scripts/seed.py}
  \item Открыть Swagger: \texttt{/docs}
\end{enumerate}

\subsection{Тестирование и устойчивость}
В ходе проверки выполнены:
\begin{itemize}[leftmargin=1.25cm]
  \item загрузка seed‑данных (1000 runs, 44000 metric values);
  \item проверка аудита: в \texttt{audit\_log} сформировано 3005 записей после seed‑загрузки;
  \item вызов функций: \texttt{fn\_best\_run\_id} возвращает корректный run\_id для выбранного эксперимента;
  \item проверка отчётов и производительности (см. \texttt{docs/perf\_report.md}).
\end{itemize}
Ошибки одиночных строк в batch‑import фиксируются без остановки всего job, что обеспечивает устойчивость обработки.

\section{Оптимизация и производительность}
Созданы индексы для полей в WHERE/JOIN/ORDER BY, включая частичный индекс для финальных метрик
(\texttt{ix\_rmv\_final\_metric}). Результаты EXPLAIN ANALYZE фиксируются в
\texttt{docs/perf\_report.md} и демонстрируют улучшение производительности до/после индексов.

Ниже приведено сравнение времени выполнения двух ключевых запросов (данные получены на seed‑наборе):

\begin{center}
\begin{longtable}{|p{7cm}|r|r|r|}
\hline
\textbf{Запрос} & \textbf{До индексов, ms} & \textbf{После индексов, ms} & \textbf{Ускорение} \\\\
\hline
Лидерборд эксперимента (fn\_experiment\_leaderboard) & 8.294 & 0.857 & 9.7x \\\\
\hline
Тренд по версиям датасетов (avg финальных метрик) & 2.496 & 1.215 & 2.1x \\\\
\hline
\end{longtable}
\end{center}

\section*{Заключение}
\addcontentsline{toc}{section}{Заключение}
Система реализует полноценную реляционную модель для управления ML‑экспериментами, поддерживает аудит,
batch‑import, аналитические отчёты и демонстрацию производительности. Решение готово к защите и соответствует
требованиям курсовой работы.

\appendix
\section{Список основных endpoints}
\begin{itemize}[leftmargin=1.25cm]
  \item CRUD: \texttt{/api/users}, \texttt{/api/orgs}, \texttt{/api/org-members}, \texttt{/api/projects},
        \texttt{/api/project-members}, \texttt{/api/datasets}, \texttt{/api/dataset-versions}, \texttt{/api/experiments},
        \texttt{/api/runs}, \texttt{/api/run-configs}, \texttt{/api/run-metric-values}, \texttt{/api/metric-definitions},
        \texttt{/api/artifacts}, \texttt{/api/run-artifacts}, \texttt{/api/audit-log},
        \texttt{/api/batch-import-jobs}, \texttt{/api/batch-import-errors}, \texttt{/api/project-metric-summary}.
  \item Отчёты: \texttt{/api/reports/experiments/{experiment\_id}/leaderboard},
        \texttt{/api/reports/experiments/{experiment\_id}/best-run},
        \texttt{/api/reports/projects/{project\_id}/dashboard}.
  \item Авторизация: \texttt{/api/auth/register}, \texttt{/api/auth/login}.
\end{itemize}

\section{Ссылки на материалы}
\begin{itemize}[leftmargin=1.25cm]
  \item Скрипты SQL: \texttt{sql/}
  \item Миграции: \texttt{migrations/}
  \item Отчёт производительности: \texttt{docs/perf\_report.md}
\end{itemize}

\end{document}
